\let\negmedspace\undefined
\let\negthickspace\undefined

\documentclass[journal,12pt,onecolumn]{IEEEtran}
%\documentclass[journal,12pt,twocolumn]{IEEEtran}
%
\usepackage{setspace}
\usepackage{gensymb}
%\doublespacing
\singlespacing

%\usepackage{graphicx}
%\usepackage{amssymb}
%\usepackage{relsize}
\usepackage[cmex10]{amsmath}
%\usepackage{amsthm}
%\interdisplaylinepenalty=2500
%\savesymbol{iint}
%\usepackage{txfonts}
%\restoresymbol{TXF}{iint}
%\usepackage{wasysym}
\usepackage{amsthm}
\usepackage{mathrsfs}
\usepackage{txfonts}
\usepackage{stfloats}
\usepackage{cite}
\usepackage{cases}
\usepackage{subfig}
%\usepackage{xtab}
\usepackage{longtable}
\usepackage{multirow}
%\usepackage{algorithm}
%\usepackage{algpseudocode}
\usepackage{enumitem}
\usepackage{mathtools}
\usepackage{tikz}
\usepackage{circuitikz}
\usepackage{verbatim}
\usepackage{hyperref}
%\usepackage{stmaryrd}
\usepackage{tkz-euclide} % loads  TikZ and tkz-base
%\usetkzobj{all}
\usepackage{listings}
\usepackage{color}                                            %%
\usepackage{array}                                            %%
\usepackage{longtable}                                        %%
\usepackage{calc}                                             %%
\usepackage{multirow}                                         %%
\usepackage{hhline}                                           %%
\usepackage{ifthen}                                           %%
%optionally (for landscape tables embedded in another document): %%
\usepackage{lscape}     
\usepackage{multicol}
\usepackage{chngcntr}
\usepackage{iftex}
%\usepackage[latin9]{inputenc}
\usepackage{geometry}
\usepackage{bm}
%\geometry{verbose,tmargin=2cm,bmargin=3cm,lmargin=1.8cm,rmargin=1.5cm,headheight=2cm,headsep=2cm,footskip=3cm}
\usepackage{array}
\newcolumntype{L}[1]{>{\raggedright\let\newline\\\arraybackslash\hspace{0pt}}m{#1}}
\newcolumntype{C}[1]{>{\centering\let\newline\\\arraybackslash\hspace{0pt}}m{#1}}
\newcolumntype{R}[1]{>{\raggedleft\let\newline\\\arraybackslash\hspace{0pt}}m{#1}}

%\usepackage{graphicx}
%\usepackage{setspace}
%\usepackage{parskip}

\def \hsp {\hspace{3mm}}

\makeatletter

\providecommand{\tabularnewline}{\\}



\makeatother
\ifxetex
\usepackage[T1]{fontenc}
\usepackage{fontspec}
%\setmainfont[ Path = fonts/]{Sanskrit_2003.ttf}
\newfontfamily\nakulafont[Script=Devanagari,AutoFakeBold=2,Path = fonts/]{Nakula}
%\newfontfamily\liberationfont{Liberation Sans Narrow}
%\newfontfamily\liberationsansfont{Liberation Sans}
\fi
\usepackage{tikz}
\usepackage{xcolor}
%\usepackage{enumerate}

%\usepackage{wasysym}
%\newcounter{MYtempeqncnt}
\DeclareMathOperator*{\Res}{Res}
%\renewcommand{\baselinestretch}{2}
\renewcommand\thesection{\arabic{section}}
\renewcommand\thesubsection{\thesection.\arabic{subsection}}
\renewcommand\thesubsubsection{\thesubsection.\arabic{subsubsection}}

\renewcommand\thesectiondis{\arabic{section}}
\renewcommand\thesubsectiondis{\thesectiondis.\arabic{subsection}}
\renewcommand\thesubsubsectiondis{\thesubsectiondis.\arabic{subsubsection}}

% correct bad hyphenation here
\hyphenation{op-tical net-works semi-conduc-tor}
\def\inputGnumericTable{}                                 %%

\lstset{
	language=tex,
	frame=single, 
	breaklines=true
}

%\begin{document}
%


\newtheorem{theorem}{Theorem}[section]
\newtheorem{problem}{Problem}
\newtheorem{proposition}{Proposition}[section]
\newtheorem{lemma}{Lemma}[section]
\newtheorem{corollary}[theorem]{Corollary}
\newtheorem{example}{Example}[section]
\newtheorem{definition}[problem]{Definition}
%\newtheorem{thm}{Theorem}[section] 
%\newtheorem{defn}[thm]{Definition}
%\newtheorem{algorithm}{Algorithm}[section]
%\newtheorem{cor}{Corollary}
\newcommand{\BEQA}{\begin{eqnarray}}
	\newcommand{\EEQA}{\end{eqnarray}}
\newcommand{\define}{\stackrel{\triangle}{=}}
\bibliographystyle{IEEEtran}
%\bibliographystyle{ieeetr}
\providecommand{\mbf}{\mathbf}
\providecommand{\pr}[1]{\ensuremath{\Pr\left(#1\right)}}
\providecommand{\qfunc}[1]{\ensuremath{Q\left(#1\right)}}
\providecommand{\sbrak}[1]{\ensuremath{{}\left[#1\right]}}
\providecommand{\lsbrak}[1]{\ensuremath{{}\left[#1\right.}}
\providecommand{\rsbrak}[1]{\ensuremath{{}\left.#1\right]}}
\providecommand{\brak}[1]{\ensuremath{\left(#1\right)}}
\providecommand{\lbrak}[1]{\ensuremath{\left(#1\right.}}
\providecommand{\rbrak}[1]{\ensuremath{\left.#1\right)}}
\providecommand{\cbrak}[1]{\ensuremath{\left\{#1\right\}}}
\providecommand{\lcbrak}[1]{\ensuremath{\left\{#1\right.}}
\providecommand{\rcbrak}[1]{\ensuremath{\left.#1\right\}}}
\theoremstyle{remark}
\newtheorem{rem}{Remark}
\newcommand{\sgn}{\mathop{\mathrm{sgn}}}
\providecommand{\abs}[1]{\left\vert#1\right\vert}
\providecommand{\res}[1]{\Res\displaylimits_{#1}} 
\providecommand{\norm}[1]{\left\lVert#1\right\rVert}
%\providecommand{\norm}[1]{\lVert#1\rVert}
\providecommand{\mtx}[1]{\mathbf{#1}}
\providecommand{\mean}[1]{E\left[ #1 \right]}
\providecommand{\fourier}{\overset{\mathcal{F}}{ \rightleftharpoons}}
%\providecommand{\hilbert}{\overset{\mathcal{H}}{ \rightleftharpoons}}
%\providecommand{\system}{\overset{\mathcal{H}}{ \longleftrightarrow}}
\providecommand{\system}[1]{\overset{\mathcal{#1}}{ \longleftrightarrow}}
\providecommand{\gauss}[2]{\mathcal{N}\ensuremath{\left(#1,#2\right)}}
%
%\newcommand{\solution}[2]{\textbf{Solution:}{#1}}
\newcommand{\solution}{\noindent \textbf{Solution: }}
\newcommand{\cosec}{\,\text{cosec}\,}
\newcommand{\sinc}{\,\text{sinc}\,}
\newcommand{\rect}{\,\text{rect}\,}
\providecommand{\dec}[2]{\ensuremath{\overset{#1}{\underset{#2}{\gtrless}}}}
\newcommand{\myvec}[1]{\ensuremath{\begin{pmatrix}#1\end{pmatrix}}}
\newcommand{\mydet}[1]{\ensuremath{\begin{vmatrix}#1\end{vmatrix}}}
\newcommand*{\permcomb}[4][0mu]{{{}^{#3}\mkern#1#2_{#4}}}
\newcommand*{\perm}[1][-3mu]{\permcomb[#1]{P}}
\newcommand*{\comb}[1][-1mu]{\permcomb[#1]{C}}
%\numberwithin{equation}{section}
\numberwithin{equation}{section}
%\numberwithin{problem}{section}
%\numberwithin{definition}{section}
\makeatletter
\@addtoreset{figure}{problem}
\makeatother
%\let\StandardTheFigure\thefigure
\let\vec\mathbf
%\renewcommand{\thefigure}{\theproblem.\arabic{figure}}
\renewcommand{\thefigure}{\arabic{section}.\arabic{figure}}
%\setlist[enumerate,1]{before=\renewcommand\theequation{\theenumi.\arabic{equation}}
	%\counterwithin{equation}{enumi}
	%\renewcommand{\theequation}{\arabic{subsection}.\arabic{equation}}
\let\StandardTheFigure\thefigure
	\vspace{3cm}
	%\usepackage{babel}
	\begin{document}
		\title{Singular Value Decomposition}
		\author{ Mannem Charan AI21BTECH11019}
		 \maketitle
		\begin{abstract}
			This report consists of my basic understanding of one of the linear algebra concepts "Singular Value Decomposition".
		\end{abstract}
                \section{Singular Value Decomposition}
                   Singular Value Decomposition, SVD is a matrix factorization technique which decomposes any matrix into three familiar matrices.It is used as a tool for data reduction in machine learning.It is often used in digital signal processing for noise reduction, image compression and etc.
		\section{Understanding SVD}
		  Let $A$ be the data matrix and what SVD tries to say that,
		   \begin{align}
			   A &= USV^T
	           \end{align}
		   where $U$ and $V$ are orthogonal matrices and $S$ is a diagonal matrix.\\
          Now we will try to see how these matrices are related to $A$, we know that 
	           \begin{align}
			 A &= USV^T \\
			 UU^T &= I\\
			 VV^T &= I
	           \end{align}
             Now consider $AA^T$,
	       \begin{align}
		       AA^T &= \brak{USV^T}\brak{VS^TU^T}\\
		            &= US^2U^T\\
		      AA^TU &= US^2
	       \end{align}
	       The last equation represents eigen vector equation of $AA^T$ with $U$ containing all the eigen vectors of $AA^T$,and with $S^2$ containing all the eigen values.Similarly if you do $A^TA$ you will get $S^2$ as the diagonal matrix containing eigen values of $A^TA$ which are same as $AA^T$ and $V$ as the matrix holding all the eigen vectors of $A^TA$.\\
	       Overall what we can conclude is $U$ is the matrix of eigen vectors of $AA^T$, $S$ is the diagonal matrix containing the square roots of eigen values of $AA^T$ $\brak{\text{$A^TA$}}$ and lastly $V$ is the matrix with eigen vectors of $A^TA$.\\
	       Using this detail, one can choose the desirable eigen vectors from the corresponding singular values $\brak{\text{square root of eigen value}}$ as these values represent variance of the data along the eigen vector.
	       \section{Questions}
	         \begin{enumerate}
		   \item What is singular value in SVD?
	           \item Into how many matrices SVD will decompose the given matrix?
	           \item What are the characteristics of the matrices after SVD?
	           \item How we will reduce the dimensions of the dataset using SVD?
		   \item Write few some applications of SVD?
	         \end{enumerate}
	       \section{Answers}
	         \begin{enumerate}
		  \item Singular value in SVD is the square root of eigen value of the corresponding vector of the data matrix.
	          \item SVD factorizes any matrix $A$ into three matrices $U$, $S$ and $V$ with relation,
			   \begin{align}
				   A &= USV^T
			   \end{align}
		  \item The three matrices $U$,$S$,$V$ have the following characteristics,
			  \begin{enumerate}
			    \item $U$ is the matrix of eigen vectors of $AA^T$
			    \item $S$ is the diagonal matrix containing the square roots of eigen values of $AA^T$
			    \item $V$ is the matrix with eigen vectors of $A^TA$
		          \end{enumerate}
		  \item The singular values in the diagonal matrix $S$ can be used to understand the amount of variance explained by each of the singular vectors.So we will select the singular vector with high singular values.By this way we reduce the dimensions by also retaining the characteristics of original data set
	          \item Below mentioned are some applications of SVD
			  \begin{itemize}
			   \item It is used to find rank of the data matrix\brak{\text{no. of non-zero singular values}}
			   \item It is widely used in image compression.
			   \item It is used to calculate various matrix operations
		          \end{itemize}
            \end{enumerate}
        \end{document}				 

